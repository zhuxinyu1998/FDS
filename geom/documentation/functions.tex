\documentclass{article}

\begin{document}

\tableofcontents

%\addcontentsline{toc}{section}{Transformations}
\section{Transformations}

%\addcontentsline{toc}{subsection}{wkb2geos}
\subsection{wkb2geos}

{\tt GEOSGeom wkb2geos(wkb* geomWKB) }

\vspace{10pt}

\noindent
The functions gets a pointer to a wkb struct and uses GEOS library to create a 
GEOSGeometry. In case {\tt geomWKB} is {\bf NULL} {\tt wkb2geos} returns {\bf NULL}.


\subsection{geos2wkb}

{\tt wkb* geos2wkb(const GEOSGeometry* geosGeometry)}

\vspace{10pt}

\noindent
The function gets a pointer to a GEOSGeometry and uses GEOS library to create a wkb struct.
In case {\tt geosGeometry} is {\bf NULL}, {\tt malloc} cannot reserve the needed memory 
or GEOS {\tt GEOSGeomToWKB\_buf} function returns {\bf NULL} the function also returns a pointer
to a {\bf NULL wkb struct}. 


\subsection{wkbFROMSTR}

{\tt int wkbFROMSTR(char* geomWKT, int* len, wkb **geomWKB, int srid)}

\vspace{10pt}

\noindent
The function is used when reading data from disk. It gets a WKT geometry representation 
{\tt geomWKT} and the {\tt srid} of it and creates the wkb struct {\tt geomWKB}. In case 
{\tt geomWKT} is {\tt str\_nil} or there is some error when creating the GEOSGeometry or 
the wkb struct from it the function creates a {\tt wkb\_nil}. It returns the number of 
characters in {\tt geomWKT} or 0 if an error occurred. {\tt *len} is set to the size of the  
{\tt geomWKB} but I am not sure why this is needed. The function returns the size of the created 
{\tt geomWKT} and it is important in order to return the correact value.


\subsection{wkbTOSTR}

{\tt int wkbTOSTR(char **geomWKT, int* len, wkb *geomWKB)}

\vspace{10pt}

\noindent
The functions is used when writting data to disk. It gets a wkb struct {\tt geomWKB} and creates
the WKT representation of it {\tt geomWKT}. If an error occurs when creating the GEOSGeometry from 
the {\tt geomWKB} or when creating the WKT from the GEOSGeometry {\tt "nil"} is returned as 
{\tt geomWKT}. {\tt *len} has the length of {\tt geomWKT} including quotes and `$\backslash$0' 
character, and is very important for the correct storage of the data.


\subsection{wkbFromText}

{\tt str wkbFromText(wkb **geomWKB, str *geomWKT, int* srid, int *tpe)}

\vspace{10pt}

\noindent
The function uses the {\tt wkbFROMSTR} to create a {\tt geomWKB} out of {\tt geomWKT}.
The function is used by the SQL functions
\begin{itemize}
\item {\tt ST\_GeomFromText}
\item {\tt ST\_GeometryFromText}
\item {\tt ST\_PointFromText}
\item {\tt ST\_LineFromText}
\item {\tt ST\_PolygonFromText}
\item {\tt ST\_MPointFromText}
\item {\tt ST\_MLineFromText}
\item {\tt ST\_MPolyFromText}
\item {\tt ST\_GeomCollFromText}
\end{itemize}
{\tt *tpe} is used to indicate the geometry type that should be created. Although all SQL
functions use the same {\tt wkbFromText} c function a check is performed whether the 
{\tt geomWKT} created a {\tt geomWKB} of the same type as {\tt *tpe}.


\section{Basic Methods on geometric objects (OGC)}

\subsection{wkbDimension}

{\tt str wkbDimension(int *dimension, wkb **geomWKB)}

\subsection{wkbGeometryType}

{\tt str wkbGeometryType(char** out, wkb** geomWKB, int* flag)}

\subsection{wkbGetSRID}

{\tt str wkbGetSRID(int *out, wkb **geomWKB)}

\subsection{Envelope}

\subsection{wkbAsText}

{\tt str wkbAsText(str *txt, wkb **geomWKB)}

\vspace{10pt}
\noindent
The function uses the {\tt wkbFROMSTR} to create a {\tt geomWKT} from {\tt geomWKB}.
The function is used by the SQL function {\tt ST\_AsText}.

\subsection{AsBinary}

\subsection{wkbIsEmpty}

{\tt str wkbIsEmpty(bit *out, wkb **geomWKB)}

\subsection{wkbIsSimple}

{\tt str wkbIsSimple(bit *out, wkb **geomWKB)}

\subsection{Is3D}

\subsection{IsMeasured}

\subsection{wkbBoundary}

{\tt str wkbBoundary(wkb **boundaryWKB, wkb **geomWKB)}

\section{Notes on sql level}

\subsection{{\tt int atom\_cast(atom *a, sql\_subtype *tp)}}
It is called to check the arguments of a function (once for each argument) \\
{\tt sql\_subtype *tp} has the type of the expected argument and {\tt sql\_subtype *at = \&a->tpe} 
has the type of the provided argument.


\end{document}
